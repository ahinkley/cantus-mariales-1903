% !TEX TS-program = LuaLaTeX+se
\documentclass[10pt]{book}
\usepackage{fontspec}
\usepackage{fancyhdr}
\usepackage{titlesec}
\usepackage[autocompile]{gregoriotex}
\usepackage[utf8]{luainputenc}
\usepackage{ebgaramond}
\usepackage{longtable}
%Litany
\usepackage{multicol}
\usepackage[savepos]{zref}
%\usepackage{libertine}
\usepackage{multicol}
\usepackage{rotating}

\usepackage[margin=0.5in, paperwidth=6in, paperheight=9in]{geometry}

\definecolor{myred}{HTML}{FF3333}
\titleformat{\chapter}{\color{myred}\normalfont\huge\bfseries\filcenter}{\color{myred}}{1em}{}
\titleformat{\section}{\color{myred}\normalfont\Large\bfseries\filcenter}{\color{myred}}{1em}{}
\titleformat{\subsection}{\color{myred}\normalfont\large\bfseries\filcenter}{\color{myred}}{1em}{}

\title{Cantus Mariales}
\author{D. Josephus Pothier, O. S. B.}
\date{23 January 1903}

\fancypagestyle{plain}{%
\fancyhead{}%
\fancyfoot[C]{-- \thepage  --}%
\renewcommand{\headrulewidth}{0pt}%
\renewcommand{\footrulewidth}{0pt}}

\pagestyle{plain}

\newfontfamily\myfont{Linux Libertine O}


%Litany code
\columnseprule=0.4pt

\makeatletter  
\newcounter{score}
\newcounter{tabstop}[score]
\newcommand{\grealign}{%
	\@bsphack%
	\ifgre@boxing\else%
		\kern\gre@dimen@begindifference%
		\stepcounter{tabstop}%
		\expandafter\zsavepos{stop-\thescore-\thetabstop}%
		\kern-\gre@dimen@begindifference%
	\fi%
	\@esphack%
}

\newcommand{\setstops}{%
  \gdef\nstabbing@stops{%
    \hspace*{-\oddsidemargin}\hspace{-1in}%
    \hspace*{\zposx{stop-\thescore-1} sp}\=%
  }%
  \count@=\@ne
  \loop\ifnum\count@<\value{tabstop}%
    \begingroup\edef\x{\endgroup
      \noexpand\g@addto@macro\noexpand\nstabbing@stops{%
        \noexpand\hspace{-\noexpand\zposx{stop-\thescore-\the\count@} sp}%
        \noexpand\hspace{\noexpand\zposx{stop-\thescore-\the\numexpr\count@+1} sp}\noexpand\=%
      }%
    }\x
    \advance\count@\@ne
  \repeat
  \nstabbing@stops\kill
}
\makeatother

\newenvironment{nstabbing}
  {\setlength{\topsep}{0pt}%
   \setlength{\partopsep}{0pt}%
   \tabbing%
   \setstops}
  {\endtabbing\stepcounter{score}}

\begin{document}
\pagenumbering{roman}
\maketitle
Transcribed from a facsimile published by Prof. Peter Kwasniewski.
Amazon links:
www.amazon.com/Cantus-Mariales-Marian-Plainchants-Ancient/dp/153273963X
www.amazon.com/Cantus-Mariales-plain-cover-Latin/dp/1532745397

\newcommand{\ts}[1]{\begin{flushright}(\textbf{\myfont T.} {\it #1})\end{flushright}}
\newcommand{\ms}[1]{\begin{flushright}(\textbf{\myfont M.} {\it #1})\end{flushright}}
\newcommand{\tms}[1]{\begin{flushright}(\textbf{\myfont T. M.} {\it #1})\end{flushright}}
\newcommand{\tmss}[2]{\begin{flushright}(\textbf{\myfont T.} {\it #1;} \textbf{\myfont M.} {\it #1})\end{flushright}}

\newfontface\berry{Berry Rotunda}
\grechangedim{annotationraise}{-8pt}{scalable}
\grechangestyle{initial}{\berry\color{red}\fontsize{24}{24}\selectfont}

\grechangestaffsize{13}

\begin{center}
\Huge{\bf CANTUS MARIALES}\\
\vspace{20pt}
\large{QUOS}\\
\vspace{20pt}
\large{E FONTIBUS ANTIQUIS ERUIT}\\
\vspace{20pt}
\large{AUT OPERE NOVO}\\
\vspace{20pt}
\large{VETERUM INSTAR CONCINNAVIT}\\
\vspace{20pt}
\huge{\bf D. JOSEPHUS POTHIER}\\
\vspace{20pt}
\large{\bf ABBAS SANCTI WANDREGISILI}\\
\vspace{20pt}
\large{\bf O. S. B.}
\end{center}

\rule{1in}{1pt}
\vfil
\begin{center}
PARISIIS\\
\vspace{10pt}
EDITIT CAR. POUSSIELGUE\\
\vspace{10pt}
15, VIA DICTA CASSETTE, 15\\
\vspace{10pt}
1903
\end{center}

IMPRIMATUR
\begin{center}
{\it Parissis, die 23 Januarii 1903}\\
\vspace{20pt}
G. LEFEBVRE,\\
Vic. gén.
\end{center}

\newpage

{\it Omnia certatim christiana sæcula gloriosissimam Virginem Dei matrem summis laudibus extulerunt. In hoc autem unanimi cinctarum generationum beatam Eam dicentium concentu præclare medium ævum suas utique partes explevit; cujus quippe ingenua ac singularis erga Reginam cœli et terræ devotio mire effloruit in multos ad ejus honorem hymnos et modulos.}

{\it Ex his omnibus sacris carminibus haud pauca certe præcellunt tum alta sententiarum doctrina, tum effusionum, tum et solerti sacræ Scripturæ figurarum et dictionum usu, quæ profanæ artis locutiones fictionesque religiosis mentibus sane superant, tum denique perpulchra rhythmorum varietate atque musicorum pia suavitate modulorum.}

{\it Hæc vero egregia pietatis avitæ monumentat jam fere cuncta in oblivionis pulvere jacent, sepulta nempe in codicibus vetustis; et sunt quasi flores eximii quos desiccatos quoddam premeret herbarium. Plurimum tamen optandum est us ex his saltem præcipua in lucem redire valeant, et ep pacto Dei laudibus et solatiis animarum aduc inservire.}

{\it Nos igitur hodie valde juvat cantus istos Mariales, qui subsequuntur, edere, partim e monumentis medii ævi feliviter effossos, partim vero ad normam veteris hujus ætatis noviter concinnatos.}

{\it Ipsi tamen hic non omnes servantur integre in primæva sua forma, scilicet ut jacent in proprio cujusque documento, quamvis optimo. Fuit enim interdum aut utile aut etiam quandoque necessarium ab originali in aliquo sive verbo, sive modulo discedere; quatenus haberetur aut rectior doctrina, aut apertior sensus, aut aptior rhythmus vel modulus. Nec mirum: præsens enim opusculum non tam ad rem antiquariuam aut litterariam quam ad precatoriam pertinet. Infra etiam pro re nata notantur, in propriis utique locis, hujusmodi variationes sive in littera sive in musica nota.}

{\it In fronte autem singulorum carminum primum inscribuntur, vice tituli proprii, verba quibus unumquodque incipit. Postea minor subditur inscriptio, qua notatur genus ad quod ea cantilena refertur, habita quidem in nuncupando ratione nominum quæ in ipsis sacræ liturgiæ libris usurpantur.}

{\it Nomina autem hæc sunt: }{\sc Canticum,} {\it si quid reperitur post singulas strophas ab omnibus communi voce respondendum;} {\sc Hymnus,} {\it si cantandæ sunt omnes strophæ sub eadem semper melodia;} {\sc Prosa} {\it vel} {\sc Sequentia,} {\it si binæ et binæ tantum pari melodia strophæ modulantur.} Sequentia {\it autem in eo differt a} Prosa {\it quod forma cantilenæ in hac contractior est vel simplicior quam in illa, item et numerus magis lege solutus. Exstant etiam nonnulli cantus ita constituti, ut simul a diversis generibus aliquid habere videantur. Tunc cui minus competit unum nomen ex iis quæ sunt propria, is communi vocabulo simpliciter} {\sc Rhythmus} {\it appellatur.}

{\it Utile prorsus atque gratum fore duximus aliquas singulis cantibus brevissimas subjungere notas, quibus præcipue variæ rerum indicarentur origines, nimirum sub hoc signo} T {\it textus, et sub isto} M {\it melodiæ fontes/ Nil vero notatur, ubi agitur de textu recentu, aut de nova melodia; neque ubi sive verba, sive moduli sunt publici juris, aut alias satis in propatulo.}

{\it Jam vero de his cantibus abunde liquet eos ese et merito et usu potissimos, qui ad sacram liturgiam propria pertinent. Illi quoque permagni sunt momenti, quos vetus vel communis recepit consuetudo. Neque tamen cæteri rejiciendi forent, eo tantum quod forte speciali approbatione carere viderentur; siquidem et ipsi, servatis servandis, licite et fructuose possunt in Ecclesiis adhiberi, saltem inter functiones minus liturgicas, at certe in conventibus sodalitatum, et ad pia prǽsertim mensis Marialis exercitia, aut alia hujusmodi. Quoties enim licet intromittere cantica in vernaculo sermone, toties a fortiori in latino, nempe in ipsa Matris Exxlesiæ lingua.}

{\it Accipiant ergo benigne ac lætanter animæ piæ hunc Marialem fasciculum ex omni florum genere, diversisque coloribus et odoramentis in horto ipsius Exxlesiæ pie collectum.}

\begin{center}
{\it O pia! quam pium est\\
Gaudere de te, Maria!\\
Et vos omnes qui diligitis Eam,\\
Ante torum, ante thronum\\
Deiparæ Virginis frequentate nobis\\
Dulcia cantica dramatis.\\
Amen.}
\end{center}


\pagenumbering{arabic}
\chapter{I\\Salutatio angelica}
\begin{center}\textcolor{red}{ANTIPHONA}\end{center}
\greannotation{\small \textcolor{red}{\textbf{6.}}}
\gregorioscore{gabc/an_salutatio_angelica.gabc}

\chapter{II\\Memorare}
\begin{center}\textcolor{red}{ANTIPHONA}\end{center}
\greannotation{\small \textcolor{red}{\textbf{1.}}}
\gregorioscore{gabc/an_memorare.gabc}

\chapter{III\\Inviolata}
\begin{center}\textcolor{red}{PROSA}\end{center}
\greannotation{\small \textcolor{red}{\textbf{6.}}}
\gregorioscore{gabc/pr_inviolata.gabc}
\ts{XII. s. Tropi mediis verbis {\bf \Rbar. Gaude Maria} intersiti.}

\chapter{IV\\O Virgo pulcherrima}
\begin{center}\textcolor{red}{PROSA}\end{center}
\greannotation{\small \textcolor{red}{\textbf{6.}}}
\gregorioscore{gabc/pr_o_virgo_pulcherrima.gabc}
\ts{Brit. Musæi, 7. A. VI. Ed. R. P. Ragey, Hymn. B. M. V., p. 29.}

\chapter{V\\Tota pulchra es}
\begin{center}\textcolor{red}{CANTICUM}\end{center}
\greannotation{\small \textcolor{red}{\textbf{5.}}}
\gregorioscore{gabc/ca_tota_pulchra_es.1.gabc}
%\greannotation{\small \textcolor{red}{\textbf{5.}}}
\gregorioscore{gabc/ca_tota_pulchra_es.2.gabc}
%\greannotation{\small \textcolor{red}{\textbf{5.}}}
\gregorioscore{gabc/ca_tota_pulchra_es.3.gabc}
%\greannotation{\small \textcolor{red}{\textbf{5.}}}
\gregorioscore{gabc/ca_tota_pulchra_es.4.gabc}
%\greannotation{\small \textcolor{red}{\textbf{5.}}}
\gregorioscore{gabc/ca_tota_pulchra_es.5.gabc}
%\greannotation{\small \textcolor{red}{\textbf{5.}}}
\gregorioscore{gabc/ca_tota_pulchra_es.6.gabc}

\ts{Centones e locis variis sacræ liturgiæ.}

\chapter{VI\\Omnis expertem}
\begin{center}\textcolor{red}{HYMNUS}\end{center}
\greannotation{\small \textcolor{red}{\textbf{4.}}}
\gregorioscore{gabc/hy_omnis_expertem.gabc}
\begin{longtable}{ll}
2. &Ipsa se præbens húmili puéllæ\\
&Virgo spectándam, récreat pavéntem\\
&Seque concéptam sine labe sancto\\
&Prædicat ore.\\
\\\\
3. &O specus felix decoráta divæ\\
&Matris aspéctu! veneránda rupes,\\
&Unde vitáles scatuére pleno\\
&Gúrgite lymphæ.\\
\\\\
4. &Huc catervátim pia turba nostris,\\
&Huc ab extérnis peregrína terris\\
&Affuit supplex, et opem poténtis\\
&Vírginis orat.\\
\\\\
5. &Excipit Mater lácrimas precántum,\\
&Donat optátam míseris salútem;\\
&Compos hinc voti pátrias ad oras\\
&Turba revértit.\\
\\\\
6. &Súpplicum, Virgo, miseráta casus,\\
&Semper o nostros réfove labóres,\\
&Impetrans mæstis bona sempitérnæ\\
&Gaudia vitæ.\\
\\\\
7. &Sit decus Patri, genitæque Proli,\\
&Et tibi compar utriúsque virtus,\\
&Spíritus semper, Deus unus, omni\\
&Témporis ævo.  Amen.
\end{longtable}


\tmss{ex Officio Apparitionis B. M. V.}{ad noc concinnata.}

\chapter{VII\\Terrena cuncta jubilent}
\begin{center}\textcolor{red}{HYMNUS}\end{center}
\greannotation{\small \textcolor{red}{\textbf{4.}}}
\gregorioscore{gabc/hy_terrena_cuncta_jubilent.gabc}
\begin{longtable}{ll}
2. &Hæc virgo verbo grávida\\
&Fit paradísi jánua,\\
&Quæ Deum mundo réddidit,\\
&Cælum nobis apéruit.\\
\\\\
3. &Felix ista puérpera\\
&Evæ lege libérrima,\\
&Quæ concépit de Spíritu,\\
&Emísit sine gémitu.\\
\\\\
4. &Dives Mariæ grémium\\
&Mundi gestávit prétium,\\
&Quo gloriamur rédimi\\
&Solúti jugo débiti.\\
\\\\
5. &Quam Patris implet Fílius,\\
&Sanctus obúmbrat Spíritus,\\
&Cælum fiunt castíssima\\
&Sanctæ puéllæ víscera.\\
\\\\
6. &Sit tibi laus Altíssime,\\
&Qui natus es ex Vírgine,\\
&Sit honor ineffábili\\
&Patri sanctóque Flámini. Amen.
\end{longtable}

\begin{longtable}{ll}
2. &Hæc virgo verbo grávida\\
&Fit paradísi jánua,\\
&Quæ Deum mundo réddidit,\\
&Cælum nobis apéruit.\\
\\\\
3. &Felix ista puérpera\\
&Evæ lege libérrima,\\
&Quæ concépit de Spíritu,\\
&Emísit sine gémitu.\\
\\\\
4. &Dives Mariæ grémium\\
&Mundi gestávit prétium,\\
&Quo gloriamur rédimi\\
&Solúti jugo débiti.\\
\\\\
5. &Quam Patris implet Fílius,\\
&Sanctus obúmbrat Spíritus,\\
&Cælum fiunt castíssima\\
&Sanctæ puéllæ víscera.\\
\\\\
6. &Sit tibi laus Altíssime,\\
&Qui natus es ex Vírgine,\\
&Sit honor ineffábili\\
&Patri sanctóque Flámini. Amen.
\end{longtable}

\ts{S. Petri Damiani, Cardinalis. Ed. Tomasi, p. 385.}

\chapter{VIII\\Eva virum}
\begin{center}\textcolor{red}{RHYTHMUS}\end{center}
\greannotation{\small \textcolor{red}{\textbf{1.}}}
\gregorioscore{gabc/rh_eva_virum.1.gabc}

%\greannotation{\small \textcolor{red}{\textbf{1.}}}
\gregorioscore{gabc/rh_eva_virum.2.gabc}

%\greannotation{\small \textcolor{red}{\textbf{1.}}}
\gregorioscore{gabc/rh_eva_virum.3.gabc}
\ts{1\textsuperscript{a} stropha. Bibl. nat. Paris.. Cod. 1139. Ed. Raillard. -- 2\textsuperscript{a} et 3\textsuperscript{a} str. novæ.}
\ms{Cod. 1139.}

\chapter{IX\\Jesu Fili summi Patris}
\begin{center}\textcolor{red}{SEQUENTIA}\end{center}
\greannotation{\small \textcolor{red}{\textbf{4.}}}
\gregorioscore{gabc/se_jesu_fili_summi_patris.gabc}
\tms{Bibl. Sangall.. Cod. 546.}

\chapter{X\\O res admirabilis}
\begin{center}\textcolor{red}{SEQUENTIA}\end{center}
\greannotation{\small \textcolor{red}{\textbf{3.}}}
\gregorioscore{gabc/se_o_res_admirabilis.gabc}
\ts{Bibl. nat. Paris.. Cod. 1339.}

\chapter{XI\\In natali Domini}
\begin{center}\textcolor{red}{CANTICUM}\end{center}
\greannotation{\small \textcolor{red}{\textbf{1.}}}
\gregorioscore{gabc/ca_in_natali_domini.gabc}
\tms{Ed. Wackernagel. {\rm Kirchenlied.} 1. 203.}

\chapter{XII\\Resonet in laudibus}
\begin{center}\textcolor{red}{CANTICUM}\end{center}
\greannotation{\small \textcolor{red}{\textbf{5.}}}
\gregorioscore{gabc/ca_resonet_in_laudibus.gabc}
\begin{longtable}{ll}
2. &Sion lauda Dóminum\\
&Salvatórem ómnium;\\
&Virgo parit Fílium.\\
* Appáruit.\\
\\\\
3. &Pueri concúrrite,\\
&Nato Regi psállite,\\
&Voce pia dícite.\\
* Appáruit.\\
\\\\
4. &Natus est Emmánuel,\\
&Quem prædíxit Gábriel\\
&Testis est Ezáchiel.\\
* Appáruit.\\
\\\\
5. &Juda cum cantóribus\\
&Grádere de fóribus,\\
&Et dic cum pastóribus.\\
* Appáruit.\\
\\\\
6. &Qui regnat in æthere,\\
&Venit ovem quærere,\\
&Nolens eam pérdere.\\
* Appáruit.\\
\\\\
7. &Et nos unanímiter\\
&Proclamémus dúlciter:\\
&Ipse summus árbiter.\\
* Appáruit.\\
\\\\
8. &Ergo nostra cóncio,\\
&Cum sit plena gáudio\\
&Benedícat Dómino.\\
* Appáruit.\\
\\\\
9. &Sancta tibi Trínitas\\
&Os ómnium grátias\\
&Résonet altíssimas.\\
* Appáruit.
\end{longtable}


\chapter{XIII\\Clemens et benigna}
\begin{center}\textcolor{red}{PROSA}\end{center}
\greannotation{\small \textcolor{red}{\textbf{2.}}}
\gregorioscore{gabc/pr_clemens_et_benigna.gabc}
\tms{Bibl. Sangall.. Cod. 546, additio \Rbar ad libitum.}

\chapter{XIV\\Stabat Mater speciosa}
\begin{center}\textcolor{red}{SEQUENTIA}\end{center}
\greannotation{\small \textcolor{red}{\textbf{6.}}}
\gregorioscore{gabc/se_stabat_mater_speciosa.gabc}
\ts{Bibl. nat. Paris.. Cod. 7785, lat. Ed. Ozanam, {\rm Poétes francisc..} Eæ tamen fuerunt merito eliminatæ strophæ quæ non videntur genuinæ, vel certe pares non habent in typica seq. {\rm Stabat mater dolorosa}, ejusdem, ut creditur, auctoris. Nonnulli etiam versus mutati sunt in alios melioris rhythmi vel sensus aptioris.}

\chapter{XV\\Ave Mater}
\begin{center}\textcolor{red}{SEQUENTIA}\end{center}
\greannotation{\small \textcolor{red}{\textbf{6.}}}
\gregorioscore{gabc/se_ave_mater.gabc}
\ts{Ed. Mone, 373}

\chapter{XVI\\Ave plena gratiæ}
\begin{center}\textcolor{red}{PROSA}\end{center}
\greannotation{\small \textcolor{red}{\textbf{6.}}}
\gregorioscore{gabc/pr_ave_plena_gratiae.gabc}
\ts{Ed. Mone, 527.}

\chapter{XVII\\Post partum Virgo}
\begin{center}\textcolor{red}{PROSA}\end{center}
\greannotation{\small \textcolor{red}{\textbf{4.}}}
\gregorioscore{gabc/pr_post_partum_virgo.gabc}
\tms{Bibl. Sangall. Cod. 546. -- Bibl. nat. Paris.. Cod. 17311, lat.}

\chapter{XVIII\\Salve mira creatura}
\begin{center}\textcolor{red}{CANTICUM}\end{center}
\greannotation{\small \textcolor{red}{\textbf{1.}}}
\gregorioscore{gabc/ca_salve_mira_creatura.gabc}
\begin{longtable}{ll}
2. &Cum creáris, absque mora\\
&Tota sancta et decóra;\\
&Non est in te nubis hora,\\
&Tota corúscans auróra\\
&Justitiæ solis.\\
\Rbar. O clemens.\\
\\\\
3. &Salve cæli plena rore,\\
&Gaudens et Matris honóre\\
&Et Virginitátis flore,\\
&Cum in lílii candóre\\
&Génitrix es facta.\\
\Rbar. O clemens.\\
\\\\
4. &Fons exúndans gaudiórum,\\
&Paris Regem Angelórum\\
&In salútem populórum,\\
&Captivórum et reórum,\\
&O mater intácta!\\
\Rbar. O clemens.\\
\\\\
5. &Eva, dum credit serpénti\\
&Mortem tulit toti genti;\\
&Sed tu, Angelo loquénti\\
&Fidens, mísere jacénti\\
&Effugásti luctum.\\
\Rbar. O clemens.\\
\\\\
6. &Tu es tellus non labóre\\
&Fecundáta vel sudóre,\\
&Sed cælésti tantum rore;\\
&Agri pleni cum odóre\\
&Germinásti fructum.\\
\Rbar. O clemens.\\
\\\\
7. &Tu excéllis, Virgo digna,\\
&Sicut cedrus inter ligna;\\
&Te præmonstrant multa signa\\
&Genitúram, o benigna,\\
&Mundi Salvatórem.\\
\Rbar. O clemens.\\
\\\\
8. &Virga te signat frondósa\\
&Aaron prodigiósa,\\
&Quæ dat valde speciósa,\\
&Miro dono pretiósa,\\
&Fructuósum florem.\\
\Rbar. O clemens.\\
\\\\
9. &O Virgínitas beáta,\\
&Mosis rubo figuráta,\\
&Cujus Prophétæ monstráta\\
&Flamma, vi sua priváta,\\
&Ardens non urébat.\\
\Rbar. O clemens.\\
\\\\
10. &Virgo mater Deo grata,\\
&Imbre cæli inundáta,\\
&Gedeónis te lanáta\\
&De supérnis irroráta\\
&Vellus prædicébat.\\
\Rbar. O clemens.\\
\\\\
11. &Salve, cælum tangens scala,\\
&A mundo repélle mala\\
&Quæ dedérunt Adæ mala:\\
&Ut plebs fac tua sub ala\\
&Sit Christo fidélis.\\
\Rbar. O clemens.\\
\\\\
12. &Fac pro tua pietáte\\
&Ut stantes in veritáte,\\
&Virtútis integritáte,\\
&Fruámur societáte\\
&Sanctórum in cælis.\\
\Rbar. O clemens.
\end{longtable}

\ts{Partim de variis locis selectus; partim de novo conditus.}

\chapter{XIX\\Flos virginum}
\begin{center}\textcolor{red}{CANTICUM}\end{center}
\greannotation{\small \textcolor{red}{\textbf{4.}}}
\gregorioscore{gabc/ca_flos_virginum.gabc}
\begin{longtable}{ll}
2. &María, Regis sólium,\\
&Dei reclinatórium,\\
&Trinitátis triclínium,\\
&Sanctitátis palátium,\\
&Vitæ reconditórium,\\
&Redemptiónis óstium.\\
\Rbar. Flos vírginum.\\
\\\\
3. &María, mater lúminis,\\
&María, mater núminis,\\
&María, fons dulcédinis,\\
&Templumque pulchritúdinis,\\
&Via beatitúdinis,\\
&Et turris fortitúdinis.\\
\Rbar. Flos vírginum.\\
\\\\
4. &María salutífera,\\
&Stella maris lucífera,\\
&Et Génitrix Deífera,\\
&Virga Jesse florígera,\\
&Oliváque fructífera,\\
&Et nardus odorífera.\\
\Rbar. Flos vírginum.\\
\\\\
5. &María melle dúlcior,\\
&Et flóribus fragrántior;\\
&María rosis grátior\\
&Et líliis candídior;\\
&Firmaménto splendídior,\\
&Et astris rutilántior.\\
\Rbar. Flos vírginum.\\
\\\\
6. &María sine mácula,\\
&Innocéntiæ régula,\\
&Et plórum præámbula,\\
&Naufragantúmque tábula:\\
&Tutéla cunctis sédula\\
&Per tot vitæ perícula.\\
\Rbar. Flos vírginum.\\
\\\\
7. &María lux fidélium,\\
&Lætítia credéntium,\\
&Terror et luctus hóstium,\\
&Consolátrix mæréntium,\\
&Sóspitas ægrotántium,\\
&Salvátrix moriéntium.\\
\Rbar. Flos vírginum.\\
\\\\
8. &María plena grátia,\\
&Plena misericórdia,\\
&Te venerántur ómnia\\
&Cæléstia, terréstria;\\
&Tu nostra spes et glória,\\
&Tu virtus et victória.\\
\Rbar. Flos vírginum.
\end{longtable}

\ts{Versus selecti ex Seq. {\rm Jubilus Virginis..} Ed. R. P. Ragey, {\rm Hymn. B. M. V.,} p. 185.}

\chapter{XX\\Sancta Maria}
\begin{center}\textcolor{red}{CANTICUM}\end{center}
\greannotation{\small \textcolor{red}{\textbf{1.}}}
\gregorioscore{gabc/ca_sancta_maria.gabc}
\begin{longtable}{ll}
2. &Casta colúmba\\
&Advenis orbi\\
&Núntia vitæ,\\
&Pacis in ore\\
&Pígnora portans.\\
&R. O pia.\\
\\\\
3. &Astra tenéntis\\
&Fília Regis\\
&Magna poténsque:\\
&Nam super omnes\\
&Tu benedícta.\\
&R. O pia.\\
\\\\
4. &Múnere divo\\
&Sola fuísti\\
&Néscia culpæ,\\
&Tota refúlgens\\
&Lúmine sancto.\\
&R. O pia.\\
\\\\
5. &Tu paradísus!\\
&Quam tua pulchra\\
&Lília vernant,\\
&Atque rosárum\\
&Púrpura splendet!\\
&R. O pia.\\
\\\\
6. &Quod rea mortis\\
&Abstulit Eva,\\
&Ecce salútis\\
&Tu reparátrix\\
&Gérmine reddis.\\
&R, O pia.\\
\\\\
7. &Clausa per Evam\\
&Jánua cæli\\
&Jam patet ad nos,\\
&Te mediánte,\\
&Rursus apérta.\\
&R. O pia.\\
\\\\
8. &Spléndida stella,\\
&Per maris undas\\
&Ne pereámus,\\
&Fulget amíca\\
&Lux tua ductrix.\\
&R. O pia.\\
\\\\
9. &Nostra dat intus\\
&Vita timóres:\\
&Cúrrimus ad te:\\
&Nunc et in hora\\
&Mortis adésto.\\
&R. O pia.
\end{longtable}

\ms{Conductus {\rm Astra tenenti}, in Ludo Danielis. Cod. Bellovac., in Bibl. Paduana. Ed. Danjou, {\rm Revue}, t. IV. v. 65.}

\chapter{XXI\\Gloria Sanctorum}
\begin{center}\textcolor{red}{RHYTHMUS}\end{center}
\greannotation{\small \textcolor{red}{\textbf{5.}}}
\gregorioscore{gabc/rh_gloria_sanctorum.gabc}
\ts{Ed. D. Gallus Morel, O. S. B.}

\chapter{XXII\\Ave Virgo speciei}
\begin{center}\textcolor{red}{CANTICUM}\end{center}
\greannotation{\small \textcolor{red}{\textbf{}}}
\gregorioscore{gabc/ca_ave_virgo_speciei.gabc}
\begin{longtable}{ll}
2. &Ave cujus partus sanctus\\
&Effugávit nostros planctus,\\
&Dum laus Deo nuntiátur\\
&In excélsis, atque datur\\
&In terris pax homínibus.\\
\Rbar. O María.\\
\\\\
3. &In te manet mel dulcórum,\\
&In te scatet fons hortórum,\\
&In te nitet decor florum,\\
&In te mulcet thus odórum,\\
&Myrrha stillat de mánibus.\\
\Rbar. O María.\\
\\\\
4. &Ave, Virgo pulchra tota,\\
&Et amícta solis rota,\\
&Atque stellis coronáta\\
&Duodénis, dum calcáta\\
&Luna tuis est pédibus.\\
\Rbar. O María.\\
\\\\
5. &Dignum nostræ laudis thema\\
&Tuum fulgens diadéma,\\
&Tot flóribus sanctitátis,\\
&Præ his rosa caritátis,\\
&Et castitátis lílio.\\
\Rbar. O María.\\
\\\\
6. &Virgo fortis, et armáta\\
&Ut ácies ordináta,\\
&Hæreses interemísti,\\
&Ac júgiter tu fuisti\\
&Christiánis auxílio.\\
\Rbar. O María.\\
\\\\
7. &O Regína, domináris\\
&Super astra, fluctus maris,\\
&Regna terræ, nec ipsórum\\
&Portas latent infernorum\\
&Tui jura regíminis.\\
\Rbar. O María.\\
\\\\
8. &Pietátem tuam cuncti\\
&Vivi noscunt et defúncti:\\
&His in flamma fer levámen;\\
&Simul illis da juvámen\\
&In hac vita certáminis.\\
\Rbar. O María.\\
\\\\
9. &Paradísi felix porta,\\
&In agóne nos confórta:\\
&Suspirámus ad te rei\\
&Nos perduc in requiéi\\
&Ætérna tabernácula.\\
\Rbar. O María.\\
\\\\
10. &Christum pro nobis exóra,\\
&Et nunc et in mortis hora,\\
&Ut nos solvet a peccátis,\\
&Et in regno claritátis\\
&Nos cóllocet per secula.\\
\Rbar. O María.
\end{longtable}

\ts{E variis locis partim depromptus, partim de novo concinnatus.}
\ms{Cf. Conductum {\rm Regis vasa deferentes}, in Ludo Danielis, ex Cod. Bellovac..}

\chapter{XXIII\\O Maria, vitæ via}
\begin{center}\textcolor{red}{CANTICUM}\end{center}
\greannotation{\small \textcolor{red}{\textbf{5.}}}
\gregorioscore{gabc/ca_o_maria_vitae_via.gabc}
\begin{longtable}{ll}
2. &Ille ductor\\
&Et instrúctor\\
&Adsit mihi Spíritus,\\
&Qui te mundam\\
&Et fecúndam\\
&Fecit esse cælitus.\\
\Rbar. O María.\\
\\\\
3. &Illa, inquam,\\
&Ne delínquam,\\
&Me consérvet grátia,\\
&Qua replévit\\
&Te qui flevit\\
&Inter tua bráchia.\\
\Rbar. O María.\\
\\\\
4. &Orbis rector,\\
&Et protéctor\\
&Noster clementíssime,\\
&Nos invíse,\\
&Ut elísæ\\
&Convaléscant ánimæ.\\
\Rbar. O María.\\
\\\\
5. &Lumen sparge\\
&Tuæ largæ\\
&Super nos cleméntiæ:\\
&Atque mæstis\\
&Da cæléstis\\
&Solámen lætítiæ.\\
\Rbar. O María.\\
\\\\
6. &Dei Fili,\\
&Ex hostíli\\
&Serva me fallácia:\\
&Da ut fiam,\\
&Per Maríam,\\
&Tua dignus grátia.\\
\Rbar. O María.\\
\\\\
7. &Ob amórem\\
&Et honórem\\
&Matris tuæ tríbue,\\
&Ut cum bonis\\
&Fruar donis\\
&Quiétis perpétuæ.\\
\Rbar. O María.\\
\\\\
8. &Te rogámus,\\
&Et laudámus,\\
&Per ipsíus méritum,\\
&Ut ætérni\\
&Nobis regni\\
&Clemens pandas áditum.\\
\Rbar. O María.
\end{longtable}

\ts{Marial. {\sc VII}, 28-36. Ed. R. P. Ragey.}

\chapter{XXIV\\Salve mater misericordiæ}
\begin{center}\textcolor{red}{CANTICUM}\end{center}
\greannotation{\small \textcolor{red}{\textbf{5.}}}
\gregorioscore{gabc/ca_salve_mater_misericordiae.gabc}
\ts{Strophæ selectæ e prolixo poemate quod gratia devotionis quidam Carmelita in usus suorum edidit.}

\chapter{XXV\\Ave maris stella, vera mellis stilla}
\begin{center}\textcolor{red}{RHYTHMUS}\end{center}
\greannotation{\small \textcolor{red}{\textbf{4.}}}
\gregorioscore{gabc/rh_ave_maris_stella_vera_mellis_stilla.gabc}
\begin{longtable}{ll}
2. &Tu rubus inústus\\
&Per quem declarátur\\
&Quod venter onústus\\
&Castus habeátur,\\
&De quo Judex justus\\
&Caste generátur.\\
&Virga Aaron flórida,\\
&Non árida,\\
&Fronde repentína\\
&De qua viror pródiit,\\
&Et éxiit\\
&Fructus vi divína,\\
&Per quem serpentína\\
&Lingua virus vómuit,\\
&Et dóluit\\
&Dolos in ruína.\\
\\\\
3. &Ave Jacob stella,\\
&Quem, dum prophetávit,\\
&Sedens in asélla\\
&Bálaam expávit;\\
&Moysis fiscélla\\
&In qua, cum se lavit,\\
&Paraónis fília\\
&Egrégia,\\
&Moysen eréxit\\
&De carécto flúminis,\\
&Qui séminis\\
&Sui spem provéxit,\\
&Dum crimen detéxit\\
&Et percússit spúrium\\
&Ægyptium\\
&Sabulóque texit.\\
\\\\
4. &Tu es mons intáctus\\
&Visus Daniéli\\
&Qui es Regi factus\\
&Augur infidéli.\\
&Lapis hinc redáctus\\
&Est Rex rector cæli,\\
&Non percússus mánibus.\\
&Homínibus\\
&Bonæ voluntátis,\\
&Tamen sine sémine\\
&De Vírgine\\
&Grata datus gratis\\
&Natus est renátis,\\
&Verbum plenum grátiæ,\\
&Rex glóriæ,\\
&Via veritátis.\\
\\\\
5. &Æsther, Assuéri\\
&Sceptrum amplexáris,\\
&Orísque sincéri\\
&Tactu osculáris.\\
&Prole Regis veri\\
&Dum tu gravidáris,\\
&Felix est compléxio\\
&Quæ vítio\\
&Caret voluptátis.\\
&Virgo, cum fit crédula,\\
&Fit bájula\\
&Floris qui non aret.\\
&Nam parit dum paret\\
&Angeli sermónibus\\
&Verácibus,\\
&Sicut res jam claret.\\
\\\\
6. &Ergo, Virgo pia,\\
&Et Mater insígnis,\\
&Tali prophetía\\
&Et tot nota signis,\\
&Me rege, María,\\
&Précibus benígnis.\\
&Educ me de cárcere\\
&Ne témere\\
&Præda sim prædónis\\
&Qui me sæpe círcuit,\\
&Et díruit\\
&Murum ratiónis.\\
&Duc me Babylónis\\
&De fornáce férrea,\\
&Sidérea\\
&Mater Salomónis.\\
&Amen.
\end{longtable}

\ts{Brit. Mus. 7. A. VI. Ed. R. P. Ragey. {\rm Hymn. B. M. V.} 387.}

\chapter{XXVI\\Ave lumen}
\begin{center}\textcolor{red}{SEQUENTIA}\end{center}
\greannotation{\small \textcolor{red}{\textbf{8.}}}
\gregorioscore{gabc/se_ave_lumen.gabc}
\ts{Bibl. nat. Paris.. Cod. 1063. Ex abbatia Arremarensi O. S. B.}

\chapter{XXVII\\Jesse proles, quibus doles}
\begin{center}\textcolor{red}{SEQUENTIA}\end{center}
\greannotation{\small \textcolor{red}{\textbf{6.}}}
\gregorioscore{gabc/se_jesse_proles_quibus_doles.gabc}
\ts{{\sc XIII\textsuperscript{0}} s. Bibl. Sangall. Cod. 383.}
\ms{Ad instar cantus de Prophetis Christi, {\rm Omnes gentes congaudentes.} Bibl. nat Paris.. Cod. 1139. {\sc XI\textsuperscript{0}} s. Edidit Ed. de Coussemaker, {\rm Histoire de l'harmonie au moyen \^age}, pl. {\sc XVIII-XXIII.}}

\chapter{XXVIII\\O Maria, mater pia}
\begin{center}\textcolor{red}{SEQUENTIA}\end{center}
\greannotation{\small \textcolor{red}{\textbf{4.}}}
\gregorioscore{gabc/se_o_maria_mater_pia.gabc}
\ts{Ed. Mone, 606.}

\chapter{XXIX\\Ave vena veniæ}
\begin{center}\textcolor{red}{RHYTHMUS}\end{center}
\greannotation{\small \textcolor{red}{\textbf{8.}}}
\gregorioscore{gabc/rh_ave_vena_veniae.gabc}
\begin{longtable}{ll}
2. &Cedrus excelléntiæ,\\
&Hyssopus amóris,\\
&Tu misericórdiæ\\
&Mater et honóris,\\
&Palma patiéntiæ,\\
&Bálsamus odóris,\\
&Melle tui roris\\
&Tu me pascas ómnibus horis.\\
\\\\
3. &Diadéma glóriæ,\\
&Lílium candóris,\\
&Myrrha pœneténtiæ,\\
&Thálamus pudóris;\\
&Sémita justítiæ,\\
&Clíbanus ardóris,\\
&Me conjúnge choris\\
&Paradísi valde decóris.\\
\\\\
4. &Caritátis scrínium,\\
&Claustrum uniónis,\\
&Amóris incéndium,\\
&Rubus visiónis;\\
&Celsum David sólium,\\
&Vellus Gedeónis,\\
&Virtúsque Samsónis,\\
&Mihi pacem da Salomónis.\\
\\\\
5. &Nardus recens, húmilis\\
&Gutta delicáta,\\
&Virga Jesse nóbilis,\\
&Fructu fecundáta,\\
&Scala Jacob stábilis\\
&Sursum eleváta,\\
&Mens sit jam purgáta\\
&Mea per te, Virgo beáta.\\
\\\\
6. &Hortus delectábilis,\\
&Arca deauráta,\\
&Terra Jesse fértilis\\
&Flore decoráta\\
&Virgo venerábilis\\
&Stirpe David nata;\\
&Per te vita data\\
&Sit et mihi pax reparáta.\\
\\\\
7. &Tu castrum tutíssimum\\
&Quod bene vallátur,\\
&Domus, aula, cívitas\\
&Quæ bene fundátur;\\
&Tu virtus, tu cháritas\\
&Quæ non minorátur;\\
&Per te concedátur\\
&Illa mihi, nec moveátur.\\
\\\\
8. &Iris nostri fœderis,\\
&Jáspidis figúra,\\
&Piáculum scéleris,\\
&Nubes non obscúra,\\
&Tuos nunquam déseris,\\
&Alma creatúra:\\
&Ad bona ventúra\\
&Trahe post te me, Virgo pura.\\
\\\\
9. &Consolátrix óptima,\\
&Tu meum solámen.\\
&Virgo prudentíssima,\\
&Fer mihi juvámen,\\
&Ut mea sic ánima\\
&Séntiat levámen,\\
&Quod post hoc certámen\\
&Mihi Deus concédat. Amen.
\end{longtable}

\ts{Bibl. nat. Paris.. Cod. 3639. Ed. R. P. Ragey, {\rm Hymn. B. M. V.}, p. 193.}

\chapter{XXX\\O tu spes mea, Maria}
\begin{center}\textcolor{red}{RHYTHMUS}\end{center}
\greannotation{\small \textcolor{red}{\textbf{5.}}}
\gregorioscore{gabc/rh_o_tu_spes_mea_maria.gabc}
\begin{longtable}{ll}
2. &Si quid mali subit mentem,\\
&Nubem fugat invadéntem\\
&Nomen tuum appellátum,\\
&Frustra nunquam invocátum:\\
&Quavis surgénte procélla\\
&Mea es amíca stella,\\
&Et ánimæ cymbam a fluctu\\
&Servas pio tuo ductu.\\
\\\\
3. &Sub tutéla tua bona,\\
&O amáta mea Patróna,\\
&Sum in vita et in morte\\
&Secúrus de mea sorte.\\
&Tu mea lux in hac via\\
&Usque portum, O María!\\
&Me vívere fac te amándo,\\
&Te amáre moriéndo.\\
\\\\
4. &Quam pulchra cordis caténa!\\
&Hac cor meum stringas amœna;\\
&Fiam cunctis vitæ horis\\
&Tui captívus amóris!\\
&Cor meum jam non est meum,\\
&Id rape, offer ad Deum;\\
&Jam ámplius id pro me nolo,\\
&Sit pro tuo Jesu solo.
\end{longtable}

\ts{Canticum {\rm O bella mia speranza}, auctore S. Alphonso de Ligorio.}
\ms{Eodem S. Doctore.}

\chapter{XXXI\\O Maria, Dei cella}
\begin{center}\textcolor{red}{PROSA}\end{center}
\greannotation{\small \textcolor{red}{\textbf{6.}}}
\gregorioscore{gabc/pr_o_maria_dei_cella.gabc}
\ts{Ad melodiam e variis locis concinnatus.}
\ms{Ex Ludo Danielis, Conductus, {\rm Jubilemus Regi nostro.}}

\chapter{XXXII\\Voce jucunditatis}
\begin{center}\textcolor{red}{SEQUENTIA}\end{center}
\greannotation{\small \textcolor{red}{\textbf{3.}}}
\gregorioscore{gabc/se_voce_jucunditatis.gabc}
\ts{Petri Venerabilis, abb. Cluniac.. Patr. lat. CXLIX.}

\chapter{XXXIII\\Concordi lætitia}
\begin{center}\textcolor{red}{RHYTHMUS}\end{center}
\greannotation{\small \textcolor{red}{\textbf{6.}}}
\gregorioscore{gabc/rh_concordi_laetitia.gabc}
\ts{Ed. Felix. Clement. Seq. 86.}
\ms{Prosa {\rm Orientis partibus}, {\sc XIII\textsuperscript{0}} s.}

\chapter{XXXIV\\Tota Regis filiæ}
\begin{center}\textcolor{red}{SEQUENTIA}\end{center}
\greannotation{\small \textcolor{red}{\textbf{4.}}}
\gregorioscore{gabc/se_tota_regis_filiae.gabc}
\ts{In Missa propria Cordis B. M. V. a S. R. C. approbata pro Societate S. Sulpitii, item SS. Cord.}

\chapter{XXXV\\Gaude, Virgo, quæ de cælis}
\begin{center}\textcolor{red}{CANTICUM}\end{center}
\greannotation{\small \textcolor{red}{\textbf{6.}}}
\gregorioscore{gabc/ca_gaude_virgo_quae_de_caelis.gabc}
\begin{longtable}{ll}
2. &Gaude, Mater Jesu Christi,\\
&Quia Virgo peperísti\\
&Creatórem ómnium.\\
\Rbar. Allelúia.\\
\\\\
3. &Gaude, per quem cornu David\\
&Stella Jacob revelávit\\
&In accéssu géntium.\\
\Rbar. Allelúia.\\
\\\\
4. &Gaude, quia resurréxit\\
&Et revíxit, et revéxit\\
&Cursor tuus brávium.\\
\Rbar. Allelúia.\\
\\\\
5. &Gaude, per quam supra chorum\\
&Sublimátur Angelórum\\
&Natúra mortálium.\\
\Rbar. Allelúia.\\
\\\\
6. &Gaude, quia te replévit,\\
&Et supra te requiévit\\
&Illustrátor córdium.\\
\Rbar. Allelúia.\\
\\\\
7. &Gaude super omnes sola,\\
&Cujus in utráque stola\\
&Complétum est gáudium.\\
\Rbar. Allelúia.\\
\\\\
8. &Tibi, Mater, supplicámus,\\
&Fac ut tecum gaudeámus\\
&In terra vivéntium.\\
\Rbar. Allelúia.
\end{longtable}

\ts{Ed. Daniel, V, 136, cum hoc titulo {\rm B. Thomæ episcopo} [Cantuar.] {\rm per Mariam revelata}.}
\ms{nova, addito \Rbar. Alleluia.}

\chapter{XXXVI\\Visne Maria}
\begin{center}\textcolor{red}{PROSA}\end{center}
\greannotation{\small \textcolor{red}{\textbf{4.}}}
\gregorioscore{gabc/pr_visne_maria.gabc}
\ts{Canticum S. Alphonsi de italico sermone in latinum versum a R. P. Franc. Xav. Reuss, Congr. SS. Redemptoris.}
\ms{nova.}

\chapter{XXXVII\\Sicut pratum picturatur}
\begin{center}\textcolor{red}{SEQUENTIA}\end{center}
\greannotation{\small \textcolor{red}{\textbf{6.}}}
\gregorioscore{gabc/se_sicut_pratum_picturatur.gabc}
\ts{{\sc XII\textsuperscript{0}} s. Bibl. nat. Paris. Cod. 3156. Edidit D. Pitra, {\rm Spicilegium Solesmense}, III.}

\chapter{XXXVIII\\Hodiernæ lux diei}
\begin{center}\textcolor{red}{SEQUENTIA}\end{center}
\greannotation{\small \textcolor{red}{\textbf{1.}}}
\gregorioscore{gabc/se_hodiernae_lux_diei.gabc}
\tms{Ex usu per medium œvum communiter recepto.}

\chapter{XXXIX\\Alma Virgo}
\begin{center}\textcolor{red}{CANTICUM}\end{center}
\greannotation{\small \textcolor{red}{\textbf{4.}}}
\gregorioscore{gabc/ca_alma_virgo.gabc}
\begin{longtable}{ll}
2. &Recordáre quod fugísti\\
&In Ægyptum corde tristi:\\
&De ténebris redde luci\\
&Nos éxsules, ac perdúci\\
&Fac ad lumen regiónis\\
&Sempitérnæ visiónis.\\
\Rbar. Alma.\\
\\\\
3. &Recordáre, Mater Christi,\\
&Die terno quem quæsísti\\
&Fílium, quem perdidísti,\\
&Quem in templo reperísti;\\
&Hunc quæréndo me sitíre\\
&Da, quæsitum inveníre\\
\Rbar. Alma.\\
\\\\
4. &Recordáre captivátum\\
&A Judæis et ligátum,\\
&Cólaphis, álapis cæsum\\
&Atque spinis coronátum;\\
&Clamórem audi cunctórum,\\
&Solve vincla peccatórum.\\
\Rbar. Alma.\\
\\\\
5. &Recordáre sublimátum\\
&Cruce, carne denigrátum;\\
&Exspirántem cum clamóre\\
&Audit orbis cum tremóre;\\
&Cujus Cruci me confíge\\
&Ac vulnéribus afflíge.\\
\Rbar. Alma.\\
\\\\
6. &Recordáre per quem planctum\\
&Corpus Jesu sacrosánctum,\\
&De Cruce pie delátum,\\
&Tenes ulnis amplexátum:\\
&Fac me, Mater, semper frui\\
&Amóris pígnore tui.\\
\Rbar. Alma.\\
\\\\
7. &Recordáre lanceátum\\
&Piis curis unguentátum:\\
&Sacro sánguine perúnge\\
&Mentem, myrrha cor compúnge\\
&In extrémis ut cum bonis\\
&Angelórum jungar thronis.\\
\Rbar. Alma.
\end{longtable}

\ts{Ed. Mone, 431. Additæ sunt strophæ 2 et 6.}
\ms{nova.}

\chapter{XL\\Salve mundi Domina}
\begin{center}\textcolor{red}{PROSA}\end{center}
\greannotation{\small \textcolor{red}{\textbf{6.}}}
\gregorioscore{gabc/pr_salve_mundi_domina.gabc}
\ts{Ed. Mone, 322.}

\chapter{XLI\\Virgo clemens}
\begin{center}\textcolor{red}{RHYTHMUS}\end{center}
\greannotation{\small \textcolor{red}{\textbf{4.}}}
\gregorioscore{gabc/rh_virgo_clemens.gabc}
\ts{Bibl. Monac. Clm., 5539. Ed. Mone, 495.}
\ms{nova.}

\chapter{XLII\\Ave, Virgo nobilis}
\begin{center}\textcolor{red}{SEQUENTIA}\end{center}
\greannotation{\small \textcolor{red}{\textbf{6.}}}
\gregorioscore{gabc/se_ave_virgo_nobilis.gabc}
\ts{Ed. Mone, 620.}
\ms{Cfr. {\rm Alleluia, \Vbar. Domine in virtute.} Dom. V. post. Pent.}

\chapter{XLIII\\Ave maris stella, clarum jubar}
\begin{center}\textcolor{red}{PROSA}\end{center}
\greannotation{\small \textcolor{red}{\textbf{4.}}}
\gregorioscore{gabc/pr_ave_maris_stella_clarum_jubar.gabc}
\ts{{\rm Parnassus Marianus}. Ed. R. P. Ragey, {\rm Hymn. B. M. V.}, p. 413.}

\chapter{XLIV\\O Sophia}
\begin{center}\textcolor{red}{PROSA}\end{center}
\greannotation{\small \textcolor{red}{\textbf{}}}
\gregorioscore{gabc/pr_o_sophia.gabc}
\ts{Cfr. {\rm Mariale} S. Anselmi. Hym. IV, str. 21-26.}
\ms{{\sc XV\textsuperscript{0}} s. Exercitam quidem et in suo saltem genere pulchram artem hæc redolet cantilena, simul vero præ singulari sonorum scala quamdam asperitatem. Modum autem musicum prænotare consulto omittimus, cum nullus sit e consuetis ad quem ea facile posset adscribi.}

\chapter{XLV\\O Maria, Mater pia, Mater}
\begin{center}\textcolor{red}{CANTICUM}\end{center}
\greannotation{\small \textcolor{red}{\textbf{3.}}}
\gregorioscore{gabc/ca_o_maria_mater_pia_mater.gabc}
\ts{Partim de variis locis selectus, partim de novo concinnatus.}
\ms{nova.}

\chapter{XLVI\\Isti sunt agni novelli}
\begin{center}\textcolor{red}{CANTICUM}\end{center}
\greannotation{\small \textcolor{red}{\textbf{6.}}}
\gregorioscore{gabc/ca_isti_sunt_agni_novelli.gabc}
\begin{longtable}{ll}
2. &O María,\\
&Mater pia,\\
&Tuum da subsídium,\\
&Quo vincámus\\
&Et vivámus\\
&In terra vivéntium.\\
\Rbar. Isti sunt.\\
\\\\
3. &O Regína,\\
&Quam divína\\
&Præelégit grátia,\\
&Cujus partus\\
&Sacrosánctus\\
&Instaurávit ómnia.\\
\Rbar. Isti sunt.\\
\\\\
4. &O quam blanda,\\
&Quam miránda\\
&Salútis remédia!\\
&Tuus Natus\\
&Immolátus\\
&Fit pro nobis hóstia.\\
\Rbar. Isti sunt.\\
\\\\
5. &Nos per sacra,\\
&Jam lavácra\\
&Renáti baptísmatis,\\
&Mox liquóre\\
&Et odóre\\
&Roborámur chrísmatis.\\
\Rbar. Isti sunt.\\
\\\\
6. &Christiánis\\
&Vitæ panis\\
&Datur in edúlium:\\
&Datur tristi\\
&Sanguis Christi\\
&Dulce refrigérium.\\
\Rbar. Isti sunt.\\
\\\\
7. &Pro tot donis,\\
&Tantis bonis\\
&Grates Deo sólvimus:\\
&Per te data,\\
&O beáta,\\
&Jure cuncta pángimus.\\
\Rbar. Isti sunt.\\
\\\\
8. &Liberári\\
&Et salvári\\
&Per te nos confídimus:\\
&Quam cleméntem\\
&Et poténtem\\
&Super omnes nóvimus.\\
\Rbar. Isti sunt.\\
\\\\
9. &Tu præclárus\\
&Es thesáurus\\
&Omnium charísmatum\\
&Sane plenus\\
&Et amœnus\\
&Hortus es arómatum.\\
\Rbar. Isti sunt.\\
\\\\
10. &Primum quidem\\
&Nobis fidem\\
&Tuis auge précibus,\\
&Et da nobis\\
&Ut te probis\\
&Imitémur áctibus.\\
\Rbar. Isti sunt.\\
\\\\
11. &Spe labéntes\\
&Firma mentes,\\
&Caritáte róbora:\\
&Fac concórdes\\
&Pelle sordes,\\
&Excúsa facínora.\\
\Rbar. Isti sunt.\\
\\\\
12. &Mater bona\\
&Nobis dona\\
&Tuum patrocínium,\\
&Ut regnémus\\
&Et laudémus\\
&Tuum semper Filium.\\
\Rbar. Isti sunt.
\end{longtable}

\ts{Strophæ 1\textsuperscript{a}, 2\textsuperscript{a}, 3\textsuperscript{a}, 8\textsuperscript{a}, 9\textsuperscript{a}, 10\textsuperscript{a}, 11\textsuperscript{a}, 12\textsuperscript{a}, ex {\rm Mariali} S. Anselmi.\\
Str. 4\textsuperscript{a}, 5\textsuperscript{a}, 6\textsuperscript{a}, 7\textsuperscript{a} novæ. \Rbar. ex sacra liturgia.}

\chapter{XLVII\\O beata Mater}
\begin{center}\textcolor{red}{CANTICUM}\end{center}
\greannotation{\small \textcolor{red}{\textbf{7.}}}
\gregorioscore{gabc/ca_o_beata_mater.gabc}
\ts{E sacra liturgia.}
\ms{Cant. {\rm Benedictus es.} Sabb. IV. Temp.}

\chapter{XLVIII\\Imperatrix Angelorum}
\begin{center}\textcolor{red}{SUPPLICATIO}\end{center}
\greannotation{\small \textcolor{red}{\textbf{6.}}}
\gregorioscore{gabc/su_imperatrix_angelorum.gabc}
\begin{longtable}{ll}
2. &Spes et salus infirmórum,\\
&Sublevátrix oppressórum\\
&Audi nos, o María!\\
\\\\
3. &Tibi, Virgo, decantántes,\\
&Tuas laudes concrepántes,\\
&Audi nos, o María!\\
\\\\
4. &In sublími sedes throno,\\
&Propulsáta precum sono,\\
&Audi nos, o María!\\
\\\\
5. &Quæ ut mater veneráris,\\
&Obtinéndo quod precáris,\\
&Audi nos, o María!\\
\\\\
6. &Præelécta sola soli\\
&Nos commenda tuæ proli,\\
&Audi nos, o María!
\end{longtable}

\ts{Ed. Mone, 585.}

\chapter{XLIX\\Ave Virgo virginum}
\begin{center}\textcolor{red}{SEQUENTIA}\end{center}
\greannotation{\small \textcolor{red}{\textbf{7.}}}
\gregorioscore{gabc/se_ave_virgo_virginum.gabc}
\ts{Ed. Mone, 368.}

\chapter{L\\O mira caritas}
\begin{center}\textcolor{red}{SEQUENTIA}\end{center}
\greannotation{\small \textcolor{red}{\textbf{5.}}}
\gregorioscore{gabc/se_o_mira_caritas.gabc}
\ts{Ex Cod. Monac. [Munich]. Clm. 5539. Ed. Mone, 365.}
\ms{Ex eodem Codice.}

\part{Appendix}
\chapter{I\\Magnificat}
\begin{center}\textcolor{red}{ANTIPHONA}\end{center}
\greannotation{\small \textcolor{red}{\textbf{6.}}}
\gregorioscore{gabc/an_magnificat.gabc}
\tms{Versus alleluiaticus in Missa Purissimi Cordis B. M. V.,\\per modum Antiphonæ decantandus.}

\chapter{II\\Vox turturis}
\begin{center}\textcolor{red}{ANTIPHONA}\end{center}
\greannotation{\small \textcolor{red}{\textbf{4.}}}
\gregorioscore{gabc/an_vox_turturis.gabc}
\tms{Versus alleluiaticus in Missa Apparitionis B. M. V. Immaculatæ,\\per modum Antiphonæ concinendus.}

\chapter{III\\Ad Virginem Dei Matrem\\Piæ salutationes}
\greannotation{\small \textcolor{red}{\textbf{1.}}}
\gregorioscore{gabc/ad_virginem_dei_matrem__piae_salutationes.gabc}
\ts{Usu receptus.}
\ms{nova.}

\chapter{IV\\Coronula Stellarum duodecim\\B. M. V.}
\greannotation{\small \textcolor{red}{\textbf{8.}}}
\input{gabc/coronula_stellarum_duodecim__b_m_v}
\ts{B. Ludovici Mariæ Grignon de Montfort. Ed. in Periodico cui titulus: Le R\`egne de Jésus par Marie. Apr. 1900.}
\ms{nova.}

\chapter{V\\Corona sacratissimi Rosarii}
%\hrule
\begin{center}
\textcolor{myred}{ANTIPHONÆ ANTE SINGULAS DECADES DECANTANDÆ}
\end{center}
%\hrule
\section{I\\Mysteria gaudiosa}

\subsection{1. Annuntiatio.}
\greannotation{\small \textcolor{red}{\textbf{1.}}}
\gregorioscore{gabc/annuntiatio.gabc}

\subsection{2. \\Visitatio.}
\greannotation{\small \textcolor{red}{\textbf{2.}}}
\gregorioscore{gabc/visitatio.gabc}

\subsection{3. Nativitas Domini.}
\greannotation{\small \textcolor{red}{\textbf{4.}}}
\gregorioscore{gabc/nativitas_domini.gabc}
\vspace{10pt}
{\it Alia antiphona:}\\
\vspace{5pt}
\greannotation{\small \textcolor{red}{\textbf{4.}}}
\gregorioscore{gabc/nativitas_domini__alia_ant.gabc}

\subsection{4. Purificatio B. M. V.}
\greannotation{\small \textcolor{red}{\textbf{8.}}}
\gregorioscore{gabc/purificatio_b_m_v.gabc}
\vspace{10pt}
{\it Alia antiphona:}\\
\vspace{5pt}
\greannotation{\small \textcolor{red}{\textbf{8.}}}
\gregorioscore{gabc/purificatio_b_m_v__alia_ant.gabc}

\subsection{5. Inventio Jesu in templo.}
\greannotation{\small \textcolor{red}{\textbf{7.}}}
\gregorioscore{gabc/inventio_jesu_in_templo.gabc}

\section{II\\Mysteria dolorosa}
\subsection{1. Agonia in horto.}
\greannotation{\small \textcolor{red}{\textbf{1.}}}
\gregorioscore{gabc/agonia_in_horto.gabc}

\subsection{2. Flagellatio.}
\greannotation{\small \textcolor{red}{\textbf{1.}}}
\gregorioscore{gabc/flagellatio.gabc}

\subsection{3. Impositio coronæ spineæ.}
\greannotation{\small \textcolor{red}{\textbf{8.}}}
\gregorioscore{gabc/impositio_coronae_spineae.gabc}

\subsection{4. Bajulatio Crucis.}
\greannotation{\small \textcolor{red}{\textbf{8.}}}
\gregorioscore{gabc/bajulatio_crucis.gabc}
\vspace{10pt}
{\it Alia antiphona:}\\
\vspace{5pt}
\greannotation{\small \textcolor{red}{\textbf{1.}}}
\gregorioscore{gabc/bajulatio_crucis__alia_ant.gabc}

\subsection{5. Crucifixio.}
\greannotation{\small \textcolor{red}{\textbf{1.}}}
\gregorioscore{gabc/crucifixio.gabc}
\vspace{10pt}
{\it Alia antiphona:}\\
\vspace{5pt}
\greannotation{\small \textcolor{red}{\textbf{3.}}}
\gregorioscore{gabc/crucifixio__alia_ant.gabc}

\section{III\\Mysteria gloriosa}
\subsection{1. Resurrectio.}
\greannotation{\small \textcolor{red}{\textbf{7.}}}
\gregorioscore{gabc/resurrectio.gabc}

\subsection{2. Ascensio.}
\greannotation{\small \textcolor{red}{\textbf{4.}}}
\gregorioscore{gabc/ascensio.gabc}

\subsection{3. Pentecostes.}
\greannotation{\small \textcolor{red}{\textbf{8.}}}
\gregorioscore{gabc/pentecostes.gabc}
\vspace{10pt}
{\it Alia antiphona:}\\
\vspace{5pt}
\greannotation{\small \textcolor{red}{\textbf{8.}}}
\gregorioscore{gabc/pentecostes__alia_ant.gabc}

\subsection{4. Assumptio B. M. V.}
\greannotation{\small \textcolor{red}{\textbf{7.}}}
\gregorioscore{gabc/assumptio_b_m_v.gabc}
\vspace{10pt}
{\it Alia antiphona:}\\
\vspace{5pt}
\greannotation{\small \textcolor{red}{\textbf{1.}}}
\gregorioscore{gabc/assumptio_b_m_v__alia_ant.gabc}

\subsection{5. Impositio diadematis B. M. V.}
\greannotation{\small \textcolor{red}{\textbf{8.}}}
\gregorioscore{gabc/impositio_diadematis_b_m_v.gabc}
\vspace{10pt}
{\it Alia antiphona:}\\
\vspace{5pt}
\greannotation{\small \textcolor{red}{\textbf{7.}}}
\gregorioscore{gabc/impositio_diadematis_b_m_v__alia_ant.gabc}

%\hrule

\chapter{VI\\Litaniæ Laurentanæ\\B. M. V.}
%\input{gabc/litaniae_laurentanae.tex}
%

\gresetinitiallines{0}

\gresetlastline{ragged}
\gregorioscore{gabc/litany/litany-01.gabc}
%\gabcsnippet{ Kyrie eléison. <i>ij.</i> Christe eléison. <i>ij.</i> Kyrie eléison. <i>ij.</i> Christe audi nos. <i>ij.</i> Christe exáudi nos. <i>ij.</i>}
\gresetlastline{justified}
\noindent
\begin{minipage}{4in}
\gregorioscore{gabc/litany/litany-02.gabc}
%\gabcsnippet{ Pater de <i>cæ</i><b>lis</b> Deus,  miserére nobis. }
\begin{nstabbing}
\>Fi-\>li \>Redémp-\>tor \>{\it mun}-\>{\bf di} \>De-\>us, \>mi-\>se-\>ré-\>re \>nobis.\\
\>Spí-\>ri-\>tus \>\>{\it San}-\>{\bf cte} \>De-\>us, \>mi-\>se-\>ré-\>re \>nobis.\\
\>San-\>cta \>Tríni-\>tas \>{\it u}-\>{\bf nus} \>De-\>us, \>mi-\>se-\>ré-\>re \>nobis.
\end{nstabbing}
\end{minipage}


\noindent
\begin{minipage}{2.5in}
\gregorioscore{gabc/litany/litany-03.gabc}
%\gabcsnippet{ Sancta María,  ora pro nobis. }
\end{minipage}
\par
\noindent
\begin{minipage}{3.0in}
\gregorioscore{gabc/litany/litany-04.gabc}
%\gabcsnippet{ Sancta Dei Génitrix,  ora pro nobis. }
\begin{nstabbing}
\>San\>cta \>Vir-\>go \>vír\>gi\>num, \>o-\>ra \>pro \>nobis. 
\end{nstabbing}
\end{minipage}
\par
\noindent
\begin{minipage}{2.5in}
\gregorioscore{gabc/litany/litany-05.gabc}
%\gabcsnippet{ Mater Christi,  ora pro nobis. }
\end{minipage}

\begin{multicols}{2}
\noindent
Mater divínæ grátiæ,\\
Mater puríssima,\hfill \begin{rotate}{270}ora pro nobis.\end{rotate} \hspace{0.2in}\\
Mater castíssima,\\
Mater invioláta,\\
Mater intemeráta,\\
Mater amábilis,\\
Mater admirábilis,\\
Mater Creatóris,\\
Mater Salvatóris,\\
Virgo prudentíssima,\\
Virgo veneránda,\\
Virgo prædicánda,\\
Virgo potens,\\
Virgo clemens,\hfill \begin{rotate}{270}ora pro nobis.\end{rotate} \hspace{0.2in}\\
Virgo fidélis,\\
Spéculum justítiæ,\\
Sedes sapiéntiæ,\\
Causa nostræ lætítiæ,\\
Vas spirituále,\\
Vas honorábile,\\
Vas insígne devotiónis,\\
Rosa mystica,\\
Turris Davídica,\hfill \begin{rotate}{270}ora pro nobis.\end{rotate} \hspace{0.2in}\\
Turris ebúrnea,\\
Domus áurea,\\
Fœderis arca,\\
Jánua cæli,\\
Stella matutína,\\
Salus infirmórum,\\
Refúgium peccatórum,\\
Consolátrix afflictórum,\\
Auxílium Christianórum,\\
Regína Angelórum,\\
Regína Patriarchárum,\\
Regína Prophetárum,\hfill \begin{rotate}{270}\\ora pro nobis.\end{rotate} \hspace{0.1in}\\
Regína Apostolórum,\\
Regína Mártyrum,\\
Regína Confessórum,\\
Regína Vírginum,\\
Regína Sanctórum ómnium,\\
Regína sine labe origináli concépta,\\
Regína sacratíssimi Rosárii,
\end{multicols}

\noindent
\begin{minipage}{4.5in}
\gregorioscore{gabc/litany/litany-06.gabc}
%\gabcsnippet{ Agnus Dei,  qui tollis peccáta mundi,  parce nobis Dómine. 
%Agnus Dei,  qui tollis peccáta mundi,  exáudi nos Dómine. 
%Agnus Dei,  qui tollis peccáta mundi,  miserére nobis. }
\end{minipage}

\subsection{Ejusdem toni cantus solemnior. }
\gresetlastline{ragged}
\gregorioscore{gabc/litany/litany-07.gabc}
%\gabcsnippet{ Kyrie eléison. <i>ij.</i> Christe eléison. <i>ij.</i> Kyrie eléison. <i>ij.</i> Christe audi nos. <i>ij.</i> Christe exáudi nos. <i>ij.</i>}
\gresetlastline{justified}

\noindent
\begin{minipage}{4in}
\gregorioscore{gabc/litany/litany-08.gabc}
%\gabcsnippet{ Pater de cælis Deus,  miserére nobis. }
\begin{nstabbing}
\>Fi-\>li \>Redémp\>tor \>{\it mun}-\>{\bf di} \>De-\>us, \>mi-\>se-\>ré-\>re \>nobis.\\
\>Spí\>ri-\>tus \>\>{\it San}-\>{\bf cte} \>De-\>us, \>mi-\>se-\>ré-\>re \>nobis.\\
\>San\>cta \>Tríni-\>tas \>{\it u}-\>{\bf nus} \>De-\>us, \>mi-\>se-\>ré-\>re \>nobis.
\end{nstabbing}
\end{minipage}

\noindent
\begin{minipage}{4in}
\gregorioscore{gabc/litany/litany-09.gabc}
%\gabcsnippet{ Sancta María  ora pro nobis.  Sancta Dei Génitrix, etc.}
\end{minipage}

\noindent
\begin{minipage}{4.5in}
\gregorioscore{gabc/litany/litany-10.gabc}
%\gabcsnippet{ Agnus Dei,  qui tollis peccáta mundi,  parce nobis Dómine. 
%Agnus Dei,  qui tollis peccáta mundi,  exáudi nos Dómine. 
%Agnus Dei,  qui tollis peccáta mundi,  miserére nobis. }
\end{minipage}

\vspace{10pt}
{\it Ad variandum cantum potest ad libitum intermisceri tonus supra positus cum alio, ut sequitur:}
\gregorioscore{gabc/litany/litany-11.gabc}
%\gabcsnippet{ Sancta María,  ora pro nobis.  Sancta Dei Génitrix,  ora pro nobis.  Sancta Virgo Vírginum,  ora pro nobis.  Mater Christi, etc.}

\vspace{10pt}
{\it Et sic prosequitur, semper alternando.}

\subsection{Cantus item festivus.}
\gresetlastline{ragged}
\gregorioscore{gabc/litany/litany-12.gabc}
%\gabcsnippet{ Kyrie eléison. <i>ij.</i> Christe eléison. <i>ij.</i> Kyrie eléison. <i>ij.</i> Christe audi nos. <i>ij.</i> Christe exáudi nos. <i>ij.</i>}
\gresetlastline{justified}

\noindent
\begin{minipage}{4.5in}
\gregorioscore{gabc/litany/litany-13.gabc}
%\gabcsnippet{ Pater de cælis Deus,  miserére nobis. }
\begin{nstabbing}
\>Fi-\>li \>Redémp-\>tor \>{\it mun}-\>{\bf di} \>De-\>us, \>mi-\>se-\>ré-\>re \>no- bis.\\
\>Spí\>ri-\>tus \>\>{\it San}-\>{\bf cte} \>De-\>us, \>mi-\>se-\>ré-\>re \>no- bis.\\
\>San\>cta \>Tríni-\>tas \>{\it u}-\>{\bf nus} \>De-\>us, \>mi-\>se-\>ré-\>re \>no- bis.
\end{nstabbing}
\end{minipage}

\noindent
\begin{minipage}{4.5in}
\gregorioscore{gabc/litany/litany-14.gabc}
%\gabcsnippet{ Sancta María,  ora pro nobis.  Sancta Dei Génitrix, etc.}
\end{minipage}
\gresetlastline{ragged}

\noindent
\begin{minipage}{4.5in}
\gregorioscore{gabc/litany/litany-15.gabc}
%\gabcsnippet{ Agnus Dei,  qui tollis peccáta mundi,  parce nobis Dómine. 
%Agnus Dei,  qui tollis peccáta mundi,  exáudi nos Dómine. 
%Agnus Dei,  qui tollis peccáta mundi,  miserére nobis. }
\end{minipage}

\vspace{10pt}
{\it Cantus variatur ad libitum, ut sequitur:}
\gregorioscore{gabc/litany/litany-16.gabc}
%\gabcsnippet{ Sancta María,  ora pro nobis.  Sancta Dei Génitrix,  ora pro nobis.  Sancta Virgo Vírginum,  ora pro nobis.  Mater Christi,  ora pro nobis, etc.}

\vspace{10pt}
{\it Possunt etiam invocationes ad libitum decantari omnes uno ac eodem modo quo supra:} Sancta Dei genitrix, ora pro nobis.

\section{Cantus simplices.}
\subsection{1. Cantus Armoricus.}
\gresetlastline{ragged}
\gregorioscore{gabc/litany/litany-17.gabc}
%\gabcsnippet{ Kyrie eléison. <i>ij.</i> Christe eléison. <i>ij.</i> Kyrie eléison. <i>ij.</i> Christe audi nos. <i>ij.</i> Christe exáudi nos. <i>ij.</i>}
\gresetlastline{justified}

\noindent
\begin{minipage}{4.5in}
\gregorioscore{gabc/litany/litany-18.gabc}
%\gabcsnippet{ Pater de cælis Deus,  miserére nobis. }
\begin{nstabbing}
\>Fi-\>li \>Redémp-\>tor \>mun-\>di \>{\it De}-\>{\bf us,} \>mi-\>se-\>ré-\>re \>nobis.\\
\>Spí-\>ri-\>tus \>\>San-\>cte \>{\it De}-\>{\bf us,} \>mi-\>se-\>ré-\>re \>nobis.\\
\>San-\>cta \>Tríni-\>tas \>u-\>nus \>{\it De}-\>{\bf us,} \>mi-\>se-\>ré-\>re \>nobis.
\end{nstabbing}
\end{minipage}
\gregorioscore{gabc/litany/litany-19.gabc}
%\gabcsnippet{ Sancta María,  ora pro nobis.  Sancta Dei Génitrix, etc.}

\noindent
\begin{minipage}{5in}
\gregorioscore{gabc/litany/litany-20.gabc}
%\gabcsnippet{ Agnus Dei,  qui tollis peccáta mundi,  parce nobis Dómine. 
%Agnus Dei,  qui tollis peccáta mundi,  exáudi nos Dómine. 
%Agnus Dei,  qui tollis peccáta mundi,  miserére nobis. }
\end{minipage}

\subsection{2. Cantus Normannicus.}
\gresetlastline{ragged}
\gregorioscore{gabc/litany/litany-21.gabc}
%\gabcsnippet{ Kyrie eléison. <i>ij.</i> Christe eléison. <i>ij.</i> Kyrie eléison. <i>ij.</i> Christe audi nos. <i>ij.</i> Christe exáudi nos. <i>ij.</i>}
\gresetlastline{justified}

\noindent
\begin{minipage}{4.5in}
\gregorioscore{gabc/litany/litany-22.gabc}
%\gabcsnippet{ Pater de cælis Deus,  miserére nobis. }
\begin{nstabbing}
\>Fi-\>li \>Redémp-\>tor \>{\it mun}-\>{\bf di} \>De-\>us, \>mi-\>se-\>ré-\>re \>nobis.\\
\>Spí-\>ri-\>tus \>\>{\it San}-\>{\bf cte} \>De-\>us, \>mi-\>se-\>ré-\>re \>nobis.\\
\>San-\>cta \>Tríni-\>tas \>{\it u}-\>{\bf nus} \>De-\>us, \>mi-\>se-\>ré-\>re \>nobis.
\end{nstabbing}
\end{minipage}
\gregorioscore{gabc/litany/litany-23.gabc}
%\gabcsnippet{ Sancta María,  ora pro nobis.  Sancta Dei Génitrix,  ora pro nobis.  Sancta Virgo Vírginum,  ora pro nobis.  Mater Christi,  ora pro nobis, etc.}

\noindent
\begin{minipage}{4.5in}
\gregorioscore{gabc/litany/litany-24.gabc}
%\gabcsnippet{ Agnus Dei,  qui tollis peccáta mundi,  parce nobis Dómine. 
%Agnus Dei,  qui tollis peccáta mundi,  exáudi nos Dómine. 
%Agnus Dei,  qui tollis peccáta mundi,  miserére nobis. }
\end{minipage}

\subsection{3. Cantus novus.}
\gresetlastline{ragged}
\gregorioscore{gabc/litany/litany-25.gabc}
%\gabcsnippet{ Kyrie eléison. <i>ij.</i> Christe eléison. <i>ij.</i> Kyrie eléison. <i>ij.</i> Christe audi nos. <i>ij.</i> Christe exáudi nos. <i>ij.</i>}
\gresetlastline{justified}
\vspace{2pt}

\noindent
\begin{minipage}{4.5in}
\gregorioscore{gabc/litany/litany-26.gabc}
%\gabcsnippet{ Pater de cælis Deus,  miserére nobis. }
\begin{nstabbing}
\>Fi-\>li \>Redémp-\>tor \>mun-\>{\it di} \>De-\>us, \>mi-\>se-\>ré-\>re \>nobis.\\
\>Spí-\>ri-\>tus \>\>San-\>{\it cte} \>De-\>us, \>mi-\>se-\>ré-\>re \>nobis.\\
\>San-\>cta \>Tríni-\>tas \>u-\>{\it nus} \>De-\>us, \>mi-\>se-\>ré-\>re \>nobis.
\end{nstabbing}
\end{minipage}

\noindent
\begin{minipage}{4.5in}
\gregorioscore{gabc/litany/litany-27.gabc}
%\gabcsnippet{ Sancta María,  ora pro nobis.  Sancta Dei Génitrix, etc.}
\end{minipage}

\noindent
\begin{minipage}{4.5in}
\gregorioscore{gabc/litany/litany-28.gabc}
%\gabcsnippet{ Agnus Dei,  qui tollis peccáta mundi,  parce nobis Dómine. 
%Agnus Dei,  qui tollis peccáta mundi,  exáudi nos Dómine. 
%Agnus Dei,  qui tollis peccáta mundi,  miserére nobis. }
\end{minipage}

\section{Cantus in forma breviori}
\subsection{I}
\gresetlastline{ragged}
\gregorioscore{gabc/litany/litany-29.gabc}
%\gabcsnippet{ Kyrie eléison,  Christe eléison,  Christe audi nos,  Christe exáudi nos.  Pater de cælis Deus,  Fili Redémptor mundi Deus,  Spíritus Sancte Deus,  miserére nobis.  Sancta María,  Sancta Dei Génitrix,  Sancta Virgo vírginum,  ora pro nobis.  Mater Christi,  Mater divínæ grátiæ,  Mater purissima,  ora.  Mater castíssima,  Mater invioláta,  Mater intemeráta,  ora.  Mater amábilis,  Mater admirábilis,  Mater Creatóris,  ora.  Mater Salvatóris,  Virgo prudentíssima,  Virgo veneránda,  ora.  Virgo prædicánda,  Virgo potens,  Virgo clemens,  ora.  Virgo fidélis,  Spéculum justítiæ,  Sedes sapiéntiæ,  ora.  Causa nostræ lætítiæ,  Vas spirituále,  Vas honorábile,  ora.  Vas insígne devotiónis,  Rosa mystica,  Turris Davídica,  ora.  Turris ebúrnea,  Domus áurea,  Fœderis arca,  ora.  Jánua cæli,  Stella matutína,  Salus infirmórum,  ora.  Refúgium peccatórum,  Consolátrix afflictórum,  Auxílium christianórum,  ora.  Regína Angelórum,  Regína Patriarchárum,  Regína Prophetárum,  ora.  Regína Apostolórum,  Regína Mártyrum,  Regína Confessórum,  ora.  Regína Vírginum,  Regína Sanctórum ómnium,  Regína sine labe origináli concépta,  ora.  Regína Sacratíssimi Rosárii,  ora. }
\gregorioscore{gabc/litany/litany-30.gabc}
%\gabcsnippet{ Agnus Dei, qui tollis peccáta mundi,  parce nobis Dómine,  exáudi nos, Dómine,  miserére nobis. }

\subsection{II}
\gregorioscore{gabc/litany/litany-31.gabc}
%\gabcsnippet{ Kyrie eléison,  Christe eléison,  Christe audi nos,  Christe exáudi nos. }
\gregorioscore{gabc/litany/litany-32.gabc}
%\gabcsnippet{ Pater de cælis Deus,  Fili Redémptor mundi Deus,  Spíritus Sancte Deus,  miserére nobis.  Sancta María,  Sancta Dei Génitrix,  Sancta Virgo vírginum,  ora pro nobis.  Mater Christi, etc.}
\gregorioscore{gabc/litany/litany-33.gabc}
%\gabcsnippet{ Agnus Dei, qui tollis peccáta mundi,  parce nobis Dómine,  exáudi nos Dómine,  miserére nobis. }

\subsection{III}
\gregorioscore{gabc/litany/litany-34.gabc}
%\gabcsnippet{ Kyrie eléison,  Christe eléison,  Christe audi nos,  Christe exáudi nos. }
\gregorioscore{gabc/litany/litany-35.gabc}
%\gabcsnippet{ Pater de cælis Deus,  Fili Redémptor mundi Deus,  Spíritus Sancte Deus,  miserére nobis.  Sancta María,  Sancta Dei Génitrix,  Sancta Virgo vírginum,  ora pro nobis.  Mater Christi, etc.}
\gregorioscore{gabc/litany/litany-36.gabc}
%\gabcsnippet{ Agnus Dei, qui tollis peccáta mundi,  parce nobis Dómine,  exáudi nos Dómine,  miserére nobis. }

\end{document}
