IMPRIMATUR
\begin{center}
{\it Parissis, die 23 Januarii 1903}\\
\vspace{20pt}
G. LEFEBVRE,\\
Vic. gén.
\end{center}

\newpage

{\it Omnia certatim christiana sæcula gloriosissimam Virginem Dei matrem summis laudibus extulerunt. In hoc autem unanimi cinctarum generationum beatam Eam dicentium concentu præclare medium ævum suas utique partes explevit; cujus quippe ingenua ac singularis erga Reginam cœli et terræ devotio mire effloruit in multos ad ejus honorem hymnos et modulos.}

{\it Ex his omnibus sacris carminibus haud pauca certe præcellunt tum alta sententiarum doctrina, tum effusionum, tum et solerti sacræ Scripturæ figurarum et dictionum usu, quæ profanæ artis locutiones fictionesque religiosis mentibus sane superant, tum denique perpulchra rhythmorum varietate atque musicorum pia suavitate modulorum.}

{\it Hæc vero egregia pietatis avitæ monumentat jam fere cuncta in oblivionis pulvere jacent, sepulta nempe in codicibus vetustis; et sunt quasi flores eximii quos desiccatos quoddam premeret herbarium. Plurimum tamen optandum est us ex his saltem præcipua in lucem redire valeant, et ep pacto Dei laudibus et solatiis animarum aduc inservire.}

{\it Nos igitur hodie valde juvat cantus istos Mariales, qui subsequuntur, edere, partim e monumentis medii ævi feliviter effossos, partim vero ad normam veteris hujus ætatis noviter concinnatos.}

{\it Ipsi tamen hic non omnes servantur integre in primæva sua forma, scilicet ut jacent in proprio cujusque documento, quamvis optimo. Fuit enim interdum aut utile aut etiam quandoque necessarium ab originali in aliquo sive verbo, sive modulo discedere; quatenus haberetur aut rectior doctrina, aut apertior sensus, aut aptior rhythmus vel modulus. Nec mirum: præsens enim opusculum non tam ad rem antiquariuam aut litterariam quam ad precatoriam pertinet. Infra etiam pro re nata notantur, in propriis utique locis, hujusmodi variationes sive in littera sive in musica nota.}

{\it In fronte autem singulorum carminum primum inscribuntur, vice tituli proprii, verba quibus unumquodque incipit. Postea minor subditur inscriptio, qua notatur genus ad quod ea cantilena refertur, habita quidem in nuncupando ratione nominum quæ in ipsis sacræ liturgiæ libris usurpantur.}

{\it Nomina autem hæc sunt: }{\sc Canticum,} {\it si quid reperitur post singulas strophas ab omnibus communi voce respondendum;} {\sc Hymnus,} {\it si cantandæ sunt omnes strophæ sub eadem semper melodia;} {\sc Prosa} {\it vel} {\sc Sequentia,} {\it si binæ et binæ tantum pari melodia strophæ modulantur.} Sequentia {\it autem in eo differt a} Prosa {\it quod forma cantilenæ in hac contractior est vel simplicior quam in illa, item et numerus magis lege solutus. Exstant etiam nonnulli cantus ita constituti, ut simul a diversis generibus aliquid habere videantur. Tunc cui minus competit unum nomen ex iis quæ sunt propria, is communi vocabulo simpliciter} {\sc Rhythmus} {\it appellatur.}

{\it Utile prorsus atque gratum fore duximus aliquas singulis cantibus brevissimas subjungere notas, quibus præcipue variæ rerum indicarentur origines, nimirum sub hoc signo} T {\it textus, et sub isto} M {\it melodiæ fontes/ Nil vero notatur, ubi agitur de textu recentu, aut de nova melodia; neque ubi sive verba, sive moduli sunt publici juris, aut alias satis in propatulo.}

{\it Jam vero de his cantibus abunde liquet eos ese et merito et usu potissimos, qui ad sacram liturgiam propria pertinent. Illi quoque permagni sunt momenti, quos vetus vel communis recepit consuetudo. Neque tamen cæteri rejiciendi forent, eo tantum quod forte speciali approbatione carere viderentur; siquidem et ipsi, servatis servandis, licite et fructuose possunt in Ecclesiis adhiberi, saltem inter functiones minus liturgicas, at certe in conventibus sodalitatum, et ad pia prǽsertim mensis Marialis exercitia, aut alia hujusmodi. Quoties enim licet intromittere cantica in vernaculo sermone, toties a fortiori in latino, nempe in ipsa Matris Exxlesiæ lingua.}

{\it Accipiant ergo benigne ac lætanter animæ piæ hunc Marialem fasciculum ex omni florum genere, diversisque coloribus et odoramentis in horto ipsius Exxlesiæ pie collectum.}

\begin{center}
{\it O pia! quam pium est\\
Gaudere de te, Maria!\\
Et vos omnes qui diligitis Eam,\\
Ante torum, ante thronum\\
Deiparæ Virginis frequentate nobis\\
Dulcia cantica dramatis.\\
Amen.}
\end{center}
